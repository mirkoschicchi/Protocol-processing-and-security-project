\chapter{Introduction}
In this chapter, we aim to give an overview of how we separated the workload and managed the project complexity.
We will describe the project implementation starting from requirements and specifications, continuing with the group composition, the roles and the related responsibilities.
Lastly, we will talk about the effort made by each group member.

\section{Problem Description}
The project required to implement a scalable and customizable simulator of inter-AS routing using the BGP protocol.
For this task, every programming language was allowed and we decided to use plain Java 13.
Specifically, we decided to implement our infrastructure from scratch, starting from layer 2 up to layer 4. The details are explained in \ref{layer2}, \ref{layer3} and \ref{layer4}.
An option that we considered, in the beginning, was to realize a dual-stack implementation to allow both IPv4 and IPv6 packets but, in the end, we didn't have enough time to do so.
Eventually, as required in the specifications, we managed to implement the BGP messages, the T-BGP and the hybrid model. Look at \ref{BGPMex} and \ref{BGPTrust}, respectively, for further information.

\section{Team Description}
Our team was composed of four members, Stephan, Filippo, Mirko and Gabriele. To introduce us, we will briefly describe our background.\newline
Stephan is a senior developer and software architect with 20 years of experience.\newline
Filippo has a bachelor degree in telecommunication engineering, while Mirko and Gabriele in computer science.\newline
All of us had some experience in product development and management.

\subsection{Roles And Responsabilities}
From the first meeting was clear that Stephan was the perfect person to organize the team and the project given his experience.\newline
Initially, we divided the roles as follow:
Stephan was assigned to the simulator creation, along with the base infrastructure. Mirko and Gabriele were in charge of the BGP messages implementation, while Filippo should have implemented the TCP and IP structures.\newline
Week by week, we changed our roles and responsibilities trying to help each other.\newline
As an example, Stephan was always called in action to solve bugs and give hints to the other members, while Gabriele concentrated mainly in the BGP update message implementation.\newline
On the other side, Filippo produced the first basic UI and implemented the trust and votes, while Mirko implemented all the BGP FSM and enhanced the UI to make it more complete and nicer.\newline
In general, there were responsibilities distinctions based on the assigned roles and all of us tried to finish our own task by the proposed deadlines.
Moreover, when possible, we helped each other to complete the tasks.

\subsection{Effort}
We decided to dedicate an average of 16 hours per week for the project implementation to respect the total hours suggested by the teacher during the first lecture.\newline
Down below, are reported the total hours spent for the whole project implementation, presentation preparation and time spent to write this document by each group member:
\begin{itemize}
  \item Stephan 114 hours
  \item Filippo 112 hours and 30 minutes
  \item Mirko 89 hours and 30 minutes
  \item Gabriele 68 hours
\end{itemize}
Down below are reported the contribution statistics made by each member in the project repository from the beginning to end of the implementation:
\begin{tabbing}
  Filippo Piconese \= \textlangle{}filippopiconese@hotmail.it\textrangle{}:\\
  insertions: \> 4570 (15\%)\\
  deletions: \> 1074 (5\%)\\
  files: \> 182 (18\%)\\
  commits: \> 47 (26\%)\\
  lines changed: \> 5644\\
  first commit: \> Sat Nov 9 19:12:36 2019 +0100\\
  last commit: \> Thu Jan 2 17:15:52 2020 +0100\\
  \\
  Gabriele Orazi \> \textlangle{}gabriorazi07@gmail.com\textrangle{}:\\
  insertions: \> 1106 (4\%)\\
  deletions: \> 2644 (13\%)\\
  files: \> 82 (8\%)\\
  commits: \> 24 (13\%)\\
  lines changed: \> 3750\\
  first commit: \> Thu Nov 7 23:41:23 2019 +0200\\
  last commit: \> Thu Jan 2 16:20:52 2020 +0100\\
  \\
  Stephan Hauser \> \textlangle{}stephan@codefreeze.ch\textrangle{}:\\
  insertions: \> 19096 (64\%)\\
  deletions: \> 15409 (76\%)\\
  files: \> 602 (60\%)\\
  commits: \> 77 (43\%)\\
  lines changed: \> 34505\\
  first commit: \> Wed Nov 6 16:10:17 2019 +0200\\
  last commit: \> Fri Jan 3 15:04:56 2020 +0100\\
  \\
  Mirko Schicchi \> \textlangle{}mirko.schicchi@gmail.com\textrangle{}:\\
  insertions: \> 4940 (17\%)\\
  deletions: \> 1028 (5\%)\\
  files: \> 139 (14\%)\\
  commits: \> 31 (17\%)\\
  lines changed: \> 5968\\
  first commit: \> Thu Nov 7 16:49:33 2019 +0100\\
  last commit: \> Thu Jan 2 18:19:16 2020 +0200\\
  \\
  total:\\
  insertions: \> 29713 (100\%)\\
  deletions: \> 20155 (100\%)\\
  files: \> 1006 (100\%)\\
  commits: \> 180 (100\%)\\
\end{tabbing}
It is worth to mention that all the members of the group have done their utmost to successfully complete the project objectives.\newline
A special thank you goes to Stephan who was able to guide us and evenly distribute the tasks taking into account our capabilities.
