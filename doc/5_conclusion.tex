\chapter{Conclusion}
To conclude, it is possible to claim that this project has been developed in a more than satisfactory way.
We have been able to follow a constant pace, finishing the project on schedule and achieving great results.
One of the keys of this success has been the organization: we tried to apply notions that we studied about \textit{Software Engineering} and, together with Stephan's \textit{Software Architecture} experience, we created a good flow during the entire development period.

The first meeting that we had it's been one of the most important ones since the made decisions over tools and environment to use. These choices helped us a lot, especially during the debugging phase. Starting from the IDE, \textit{IntelliJ} is maybe the best setup to use when you need to work with Java. Debugging exchange of packets that we created it's been easier and pretty efficient thanks to the usage of PCAP files, allowing us to use Wireshark to analyze the correctness of the form and the content of the packet itself.
Tunnelling virtual devices it's been also pretty useful, as well as the usage of proper logging system to analyze each node and its behaviours. We also gain the most from the GUI, which has not been just something to submit together with the core software, but a real and powerful debugging tool able to widely simplify debugging and development in general.
Apart from the used tools, we found particularly appropriate to use \textit{reactive programming} from the very beginning. After the development, we are happy to have chosen Java over C++ as a programming language, as well as using JavaFX over the Swing framework for the GUI realization.
Despite the good scheduling and tasks division, some decisions took up quite a bit of our time. We always tried to simplify as much as we could the implementation, but in some cases that revealed to be counter-productive. Thanks to the over-simplification made in our TCP implementation, we were forced to do some workarounds to be compliant with the BGP state machine. In the same way, we spent a considerable amount of time to be able to restart the communication after a connection loss due to BGP collision avoidance. Other small but difficult fixes have been necessary because of the too simplified node model design in the very beginning.

In the end, we can certainly say that the project was engaging and we are sure we made the right choice working on this project rather than on the learning journal. We learned a lot from a technical point of view and teamwork coordination.
