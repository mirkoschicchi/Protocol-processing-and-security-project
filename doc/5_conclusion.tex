\chapter{Conclusion}
%Conclusion (how did you succeed; what would you do differently if you would start the project now, any other concluding remarks about the project or the course)

So far it's possible to claim that this project has been developed in a more than satisfactory way.
We've been able to follow a constant pace, finishing the project on schedule and achieving great results.
One of the keys of this success has been the organization: we tried to apply notions that we studied about \textit{Software engineering} and, together with the Stephan's \textit{Architecture experience}, we created a good flow during the entire development period.

The first meeting that we had it's been one of the most important one, since the made decisions over tools and environment to use. This choices helped us a lot, especially during the debugging phase. Starting from the IDE, \textit{IntelliJ} is maybe the best setup to use when you need to work with Java. Debugging exchange of packets that we created it's been easier and pretty efficient thanks to the usage of PCAP files, allowing us to use Wireshark to analyze the correctness of the form and the content of the packet itself.
Tunnelling virtual devices it's been also pretty useful, as well as the usage of proper logging system to analyze each node and its behaviors. We also gain the most from the GUI, which it's been not just something to be submitted together with the core software, but a real and powerful debugging tool able to widely simplify debugging and development in general.
A part from the used tools, we found particularly appropriate to use \textit{reactive programming} from the very beginning. At the conclusion of the development, we are happy to have chosen Java over C++ as programming language, as well as using JavaFX over the Swing framework for the GUI realization.
Despite the good scheduling and division of the tasks, some decisions have taken up quite a bit of our time. We always tried to simplify as much as we could, but in some cases that has been revealed to be counter-productive. Thanks to the over-simplification made in our TCP implementation, we were forced to do some workarounds in order to be compliant with the BGP state machine. In the same way, we spent quite a lot time to been able to restart the communication after a connection loss due to the BGP collision avoidance. Other small but difficult fixes has been necessary because of the too simply node model designed in the very beginning.

In the end, we can certainly say that the project was a lot of fun and we are sure we made the right choice working on this project rather than on the learning journal. We learned a lot either from a technical point of view and from teamwork coordination.