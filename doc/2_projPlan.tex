\chapter{Project Planning}\label{projPlan}
The kick-off meeting was very important to make a plan for the project and take the most important decisions concerning the implementation and development of the code.
First of all, we agreed on using plain Java 13 (without the use of a simulator) because we wanted to implement all the protocols and packets from scratch to have a deeper understanding of the proposed task.\\
We also determined to implement IPv6 as a network layer protocol since it has simpler fields concerning IPv4 and it's very important to get a better knowledge of it in the prospect of future widespread use. Besides, we were also open to doing a dual-stack implementation in case we would have enough time (but it was not the case).

Thanks to our software engineering experience we were able to subdivide the main task into subtasks and we put milestones and preferred deadlines for each of them, trying to predict the number of working hours required.\\
The work was split between the team members in the following way:
\begin{itemize}
 \item Simulator and infrastructure, Layer 2, Project management (Stephan)
 \item IPv6 (excluding routing) and TCP (Filippo)
 \item Routing and BGP (Mirko and Gabriele)
\end{itemize}

The first main step was to define almost all the interfaces that we would need to implement.
In this way, we could have a skeleton of the final product and we were able to call functions or create objects that in reality were not implemented yet.
Moreover, almost every class has been accompanied by some JUnit tests to verify the correctness of the implementation and perform bug fixing module by module.

We managed to follow quite well the plan that we did at the beginning but we slightly adapted it from time to time.
Very important for future debugging purposes was the implementation of a PCAP exporter to check the correctness of created packets using Wireshark.

\section{Weeks 1 \& 2}
At the end of the second week of work these were the results:

\begin{itemize}
 \item Simulation infrastructure (done)
 \item Layer 2 simulation based on Ethernet (done)
 \item PCAP export for all traffic (done)
 \item IPv6 packet parser \& generator (done)
 \item TCP datagram parser \& generator (done)
 \item BGP4 message parser \& generator (70 \%)
 \item IP routing table \& routing (60 \%)
\end{itemize}

For week 3 we planned to implement end-to-end connection using TCP, dynamic routing using BGP and the possibility of loading custom based scenarios.
To go more into the details:

\begin{itemize}
 \item Simulator: Load custom scenarios from files
 \item Simulator: Allow possibility to enable/disable links and nodes at runtime
 \item Finish IP routing (week 2 task)
 \item TCP server and client implementation
 \item BGP state machine
 \item BGP open, update, notification message handlers
\end{itemize}

\section{Week 3}
In this week Filippo joined Stephan in the implementation of the simulator and core networking.

The results obtained at the end of the week are the following:

\begin{itemize}
 \item Bug fixing of previous tasks (done)
 \item BGP4 message parser \& generator (80 \%)
 \item implementation of longest prefix match algorithm (done)
 \item Simulator: Load customer scenarios from files (done)
 \item Simulator: Allow possibility to enable/disable links and nodes at runtime (moved to next week)
 \item TCP server \& client implementation (20 \%)
 \item BGP state machine (50 \%)
 \item BGP open, update, notification message handlers (moved to next week)
 \item Static routing (done)
\end{itemize}

For week 4 we planned to finish BGP based routing and the basic simulator interface.
In particular we planned to:
\begin{itemize}
 \item Finish previous week tasks:
 \begin{itemize}
 \item BGP message parser
 \item PATH attribute
 \item Simulator: allow to enable/disable links and nodes at runtime
 \item BGP open, update, notification message handlers
 \item TCP server and client implementation
 \end{itemize}
 \item BGP state machine
 \item Implement a basic interface for the network and routers
 \item Wire everything together
\end{itemize}

\section{Week 4}
During the fourth week of work, we finished the BGP base routing and implemented the basic simulator interface. Going into the details:
\begin{itemize}
 \item PATH attribute for UPDATE message (done)
 \item Simulator: Allow to enable/disable links and nodes at runtime (done)
 \item BGP open, update, notification message handlers (scheduled during the weekend)
 \item TCP server and client implementation (done)
 \item BGP state machine ($80\%$)
 \item Implement a basic interface for the network and routers (done)
 \item Wire everything together (scheduled for the weekend)
\end{itemize}

For the next two weeks, we planned to implement error handling, BGP trust routing and voting. In particular:
\begin{itemize}
 \item Finish the tasks of the previous week during the weekend
 \item Improve simulation interface by showing runtime information
 \item BGP error handling
 \item BGP trust
 \item BGP voting
\end{itemize}

\section{Weeks 5 \& 6}
In these two weeks, we concluded the project, by implementing all the missing features and cleaning the code from bugs. In the end, we performed a deep refactoring process.\\
In conclusion, we can say that we managed to arrange pretty well the tasks that need to be done every week and that it was essential to organize the work beforehand, to have a scheme to follow and an idea about the amount of effort needed.
