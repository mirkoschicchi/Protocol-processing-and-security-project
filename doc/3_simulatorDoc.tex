\chapter{Simulator Documentation}\label{simulatorDoc}
We dedicate this part of the report to explain how to download and use our simulator, along with his features.\newline
First of all, the user should download the repository as follow:
\begin{verbatim}
  git clone https://github.com/mirkoschicchi/Protocol-processing-and-security-project.git
\end{verbatim}

\section{Project Structure}
This project is created with Maven and, as said in chapter \ref{projPlan}, Java 13 is needed to run the code.\newline
a quick overview of the project is reported in the following points:
\begin{itemize}
  \item The repository folder is divided into "doc", which contains the files related to this report, and "protsim", where the simulator files lie;
  \item Inside protsim/src/main/java/fi/utu/protproc/group3 you can find the simulator components like the configuration files, the UI, the nodes definition and the structures used to exchange data;
  \item Inside protsim/src/test/java/fi/utu/protproc/group3 there are the unit tests created to continuously check the code correctness upon modifications;
  \item In the folder protsim/scenarios there are the .yaml files needed to communicate to the simulator the network structure;
  \item There are other folders not needed for the project overview;
\end{itemize}

\section{Usage}
You can use any Java IDE like Intellij Idea or Eclipse.
Navigate the folders untill you reach the App.java file in the main folder. It contains the starting function of the simulator.\newline
To run the code is required to specify as a parameter one of the simulation scenarios contained in the scenarios folder (protsim/scenarios):
\begin{verbatim}
  run [-q] [-n=<network>] [-w=<networkFile>] <scenarioFile>
\end{verbatim}
The other options that you can see above are:
\begin{verbatim}
  -q, --no-gui              Disables the simulation UI
  -n, --network=<network>   Limit traffic capture to the given network
  -w, --write=<networkFile> Write network traffic to file (ngpcap file format)
\end{verbatim}
Some usage examples can be:
\begin{verbatim}
  run -w=/tmp/protsim.ngpcap -n=net1 scenarios/static.yaml (static routing example)
  run -w=/tmp/protsim.ngpcap -n=t1 scenarios/simple.yaml
  run -w=/tmp/protsim.ngpcap -n=t1 -q scenarios/demo.yaml
\end{verbatim}
